[1] Prabhakar, Nirmit, "Direct Adaptive Control for a Trajectory Tracking UAV" (2018). Doctoral Dissertations and Master's Theses. 431.
https://commons.erau.edu/edt/431
[2]M. Liu, X. Dong, Q. Li and Z. Ren, "Model Reference Adaptive Control of a Quadrotor UAV based on RBF Neural Networks," 2018 IEEE CSAA Guidance, Navigation and Control Conference (CGNCC), Xiamen, China, 2018, pp. 1-6, doi: 10.1109/GNCC42960.2018.9019021.
[3]Flores, G.R., Flores, A., & Oca, A.M. (2019). A full controller for a fixed-wing UAV. ArXiv, abs/1903.03945.
[4]Z. Zuo, C. Liu, Q. -L. Han and J. Song, "Unmanned Aerial Vehicles: Control Methods and Future Challenges," in IEEE/CAA Journal of Automatica Sinica, vol. 9, no. 4, pp. 601-614, April 2022, doi: 10.1109/JAS.2022.105410.
[5]A. Brezoescu, T. Espinoza, P. Castillo, R. Lozano, Adaptive trajectory following for a fixed-wing UAV in presence of crosswind, J. Intelligent & Robotic Systems 69 (2013) 257-271.
[6]Salazar, Sergio, Ivan Gonzalez-Hernandez, Ricardo Lopez, and Rogelio Lozan, Simulationand robust trajectory-tracking for a Quadrotor UAV. Unmanned Aircraft Systems (ICUAS),2014 International Conference on. IEEE, 2014. 
[7]Nasab, H. M., & Navazani, N. (2016). Adaptive Control for Trajectory Tracking of an Unmanned Aerial Vehicle. In Advanced Engineering Forum (Vol. 17, pp. 101–110). Trans Tech Publications, Ltd. https://doi.org/10.4028/www.scientific.net/aef.17.101
[8]Anuradha M. Annaswamy, Alexander L. Fradkov, A historical perspective of adaptive control and learning, Annual Reviews in Control, Volume52, 2021, Pages 18-41, ISSN1367-5788,
https://doi.org/10.1016/j.arcontrol.2021.10.014. 
[9]Valavanis, K.P.; Vachtsevanos, G.J. Handbook of Unmanned Aerial Vehicles; Springer: Berlin/Heidelberg, Germany, 2015; Vol. 1.
[10]S. Kamal, A. Chalanga, Ramesh Kumar P. and B. Bandyopadhyay, "Multivariable continuous integral sliding mode control," 2015 International Workshop on Recent Advances in Sliding Modes (RASM), Istanbul, Turkey, 2015, pp. 1-5, doi: 10.1109/RASM.2015.7154646.